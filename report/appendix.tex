\appendixpage
\section{代码}
\subsection{算法代码}
\begin{verbatim}
from __future__ import division  # 将整数/整数=整数的除法改为真正的除法
from PIL import Image  # 用于处理图像
import matplotlib.pyplot as plt  # 用于处理绘图
import numpy as np  # 用于操作矩阵
from utility import normalize_gray as normalize  # !在此选择归一化函数

# 全局参数
N = 40  # 图片总类数
Pn = 10  # 每类图片的总数
Pi = 5  # 每类图片训练集张数
Pj = Pn - Pi  # 每类图片的预测集张数

#! pillow中图像size表示是2-tuple: (width, height)
#! AT&T:(92, 112)
downsampling_sizes = [(round(92/40), round(112/40)), (round(92/35), round(112/35)), 
    (round(92/30), round(112/30)), (round(92/25), round(112/25)), (round(92/20), round(112/20)), 
    (round(92/14), round(112/14)), (round(92/10), round(112/10)), (round(92/5), round(112/5))]  # 不同轮次的重采样数据
sampling_type = Image.LANCZOS  # 重采样类型

overall_info = {}  # 不同轮次的结果存储

# 以不同重采样尺寸循环运行
for downsampling_size in downsampling_sizes:

    # 循环实验参数
    predictor = []  # 用于存储predictor的数组
    info = {'total_recognition_accuracy': 0.}  # 存储结果

    # 构建N个带有Pi个图片的predictor的数组
    i = 0
    while i < N:
        j = 0
        path1 = "orl_faces\\s"+str(i+1)+"\\"  # 第一层路径循环文件夹
        ims = []  # 初始化临时存储图片变量
        # 读取一组图片到list中
        while j < Pi:
            # 读取一张图片
            path = path1+str(j+1)+".pgm"  # 第二层路径循环图片文件
            im = Image.open(path)
            im.show()
            im = im.resize(downsampling_size, sampling_type)  # 重采样
            # 将该图片插入到该类数组中
            ims.append(list(im.getdata()))
            # 标准化
            normalize(ims[j])
            j += 1
        # 将ims转换为predictor并插入到数组中
        p = np.matrix(ims[0])  # 先插入头一个图片
        j = 1  # 此时j已经有一个
        # 将剩下的插入临时predictor
        while j < Pi:
            p = np.vstack((p, ims[j]))
            j += 1
        # 将predictor转置之后插入list
        predictor.append(p.T)
        i = i+1

    # 创建一个N*Pj大小的存储隶属类别的数组
    inner = []
    result = []
    # 创建一个待拷贝的长度为Pj的内链表
    for i in range(Pj):
        inner.append(0)
    # 以内链表进行拷贝创建隶属类别链表
    for i in range(N):
        result.append(inner.copy())

    # 将所有测试图片求取隶属类别
    i = 0
    while i < N:  # 外循环类别
        path1 = "orl_faces\\s"+str(i+1)+"\\"
        j = 0
        while j < Pj:  # 内层循环图片
            path = path1+str(j+Pi+1)+".pgm"  # 跳过前面Pi张训练图片
            test_im = Image.open(path)
            test_im = test_im.resize(downsampling_size, sampling_type)  # 重采样
            y = list(test_im.getdata())
            # 对y进行标准化
            normalize(y)
            # 转化为matrix并转置
            y = np.matrix(y).T
            dis = []
            for p in predictor:
                beta = np.dot(np.linalg.inv(np.dot(p.T, p)), np.dot(p.T, y))
                yhat = np.dot(p, beta)
                dis.append(np.sqrt(np.sum(np.square(y-yhat))))
            # 找到最小距离所在的索引位置,将之视作该图片从属的类别,记录之
            result[i][j] = dis.index(min(dis))+1
            j += 1
        i += 1
    # print(result)

    recognition_accuracy = []
    # 处理识别结果
    type_index = 1
    for type in result:
        recognition_accuracy.append(0.)
        for rec in type:
            if rec == type_index:
                recognition_accuracy[type_index-1] += 1
        recognition_accuracy[type_index -1] = recognition_accuracy[type_index-1]/Pj
        info['total_recognition_accuracy'] += recognition_accuracy[type_index-1]
        type_index += 1
    info['total_recognition_accuracy'] /= N  # 平均识别率
    # print("平均识别率:", recognition_accuracy)
    # print("总识别率:", info['total_recognition_accuracy'])
    run_key = int(downsampling_size[0]*downsampling_size[1])
    overall_info[run_key] = info['total_recognition_accuracy']

print(overall_info)
\end{verbatim}
\subsection{图表输出}
\begin{verbatim}
# 不同设置下算法表现折线图
r_box = ...
r_lanczos = ...
r_nearest = ...
r_box_minmax_norm = ...
r_lanczos_minmax_norm = ...
r_nearest_minmax_norm = ...

fig = plt.figure()

plt.plot(r_box.keys(),r_box.values(),'o-', label='BOX+256')
plt.plot(r_box_minmax_norm.keys(),r_box_minmax_norm.values(),'bo--', label='BOX+data')
plt.plot(r_lanczos.keys(),r_lanczos.values(),'*-', 
label = 'LANCZOS+256')
plt.plot(r_lanczos_minmax_norm.keys(),r_lanczos_minmax_norm.values(),'c*--', label = 'LANCZOS+data')
plt.plot(r_nearest.keys(),r_nearest.values(),'^-', 
label = 'NEAREST+256')
plt.plot(r_nearest_minmax_norm.keys(),r_nearest_minmax_norm.values(),'g^--', label = 'NEAREST+data')

plt.xlabel('Feature Dimension')
plt.ylabel('Recognition Accuracy')
plt.title('Recognition Accuracy with Respect to Feature Dimension')
plt.axis([0,400,0,1])

plt.legend()
plt.grid()
plt.savefig("RA_Method.pdf",format = "pdf")
plt.show()

# 使用表现最好的设置选项输出的每一组结果)
# 原始结果
i_result = ...
# 求出每组的成功率
i_recognition_accuracy = []
i_type_index = 1
for type in i_result:
    i_recognition_accuracy.append(0.)
    for i_rec in type:
        if i_rec == i_type_index:
            i_recognition_accuracy[i_type_index-1] += 1
    i_recognition_accuracy[i_type_index -1] = i_recognition_accuracy[i_type_index-1]/Pj
    i_type_index += 1
print(i_recognition_accuracy)

xl = np.arange(1,N+1)

plt.bar(xl,i_recognition_accuracy)
plt.xlabel('Picture Class Number')
plt.ylabel('Recognition Accuracy')
plt.title("Recognition Accuracy of Each Class")
plt.grid(axis = 'y')
plt.savefig('RA_Class.eps',format = 'eps')
plt.show()
\end{verbatim}


\section{详细实验数据}
\begin{table}[htbp]
    \centering
    \caption{不同设置下的LRC识别精度差异——详细实验结果}\label{tab-detailed_results}
    \begin{tabular}{ccccccccc}
        \toprule
        & \multicolumn{8}{c}{\textbf{Feature dimension}} \\
        \cmidrule{2-9}
        & 6 & 9 & 12 & 16 & 30 & 56 & 99 & 396 \\
        \midrule
        NEAREST+data & 0.07 & 0.22 & 0.375 & 0.525 & 0.680 & 0.805 & 0.845 & 0.870 \\
        \midrule
        NEAREST+256 & 0.05 & 0.240 & 0.375 & 0.605 & 0.710 & 0.820 & 0.850 & 0.875 \\
        \midrule
        BOX+data & 0.085 & 0.580 & 0.750 & 0.830 & 0.885 & 0.915 & 0.915 & 0.910 \\
        \midrule
        BOX+256 & 0.160 & 0.710 & 0.790 & 0.845 & 0.890 & \textbf{0.925} & 0.920 & 0.905 \\
        \midrule
        LANCZOS+data & 0.065 & 0.585 & 0.770 & 0.835 & 0.894 & 0.905 & 0.910 & 0.900 \\
        \midrule
        LANCZOS+256 & 0.100 & 0.675 & 0.815 & 0.885 & 0.920 & 0.915 & 0.915 & 0.905 \\
        \bottomrule
    \end{tabular}
\end{table}

